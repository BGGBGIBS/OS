\begin{table}[htbp]
    \centering
    \begin{tabularx}{\textwidth}{
        >{\raggedright\arraybackslash}p{2cm} 
        >{\raggedright\arraybackslash}X 
        >{\raggedright\arraybackslash}X}
        \toprule
        \rowcolor[HTML]{EFEFEF}
        \textbf{}           & \textbf{TLS}                                                                      & \textbf{Chiffrement par clé} \\
        \midrule
        Nature              & Protocole de sécurité pour les communications réseau                              & Méthode pour chiffrer/déchiffrer des données \\
        \midrule
        Utilisation de clés & Utilise à la fois des clés symétriques et asymétriques                            & Peut utiliser des clés symétriques ou asymétriques \\
        \midrule
        Objectif principal  & Sécuriser des sessions ou connexions entières                                     & Sécuriser des données ou messages spécifiques \\
        \midrule
        Mise en œuvre       & Négocie les paramètres de chiffrement avant le transfert de données               & Appliqué aux données elles-mêmes, indépendamment du canal \\
        \midrule
        Applications        & Sécurisation du web (HTTPS), courrier électronique, messagerie instantanée, etc.  & Chiffrement de fichiers, communication sécurisée, authentification, signature numérique \\
        \bottomrule
    \end{tabularx}
    \caption{Comparaison entre \acrshort{tls} et le chiffrement par clé}
    \label{tab:TLSvsKeyEncryption}
\end{table}