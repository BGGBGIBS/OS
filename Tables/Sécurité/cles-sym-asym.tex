% \begin{table}[htbp]
%     \centering
%     \begin{tabularx}{\textwidth}{
%         >{\raggedright\arraybackslash}p{2cm} 
%         >{\raggedright\arraybackslash}p{6.5cm} 
%         >{\raggedright\arraybackslash}p{6.5cm}}
%         \toprule
%         \rowcolor[HTML]{EFEFEF}
%         \textbf{}               & \textbf{Clés Symétriques}                                                                                                                         & \textbf{Clés Asymétriques} \\
%         \midrule
%         Méthode de chiffrement  & Utilise la même clé pour le chiffrement et le déchiffrement, ce qui rend le processus plus direct mais nécessite une gestion sécurisée de la clé. & Utilise une paire de clés: une publique pour le chiffrement et une privée pour le déchiffrement, facilitant la distribution des clés mais augmentant la complexité. \\
%         \midrule
%         Vitesse                 & Plus rapide due à des opérations moins complexes, idéale pour le chiffrement de grands volumes de données.                                        & Plus lent à cause des opérations cryptographiques plus complexes, adapté pour le chiffrement d'informations sensibles en petites quantités. \\
%         \midrule
%         Gestion des clés        & Nécessite une méthode sécurisée pour partager la clé entre les parties, moins adapté pour les systèmes à grande échelle.                          & La gestion des clés est simplifiée puisque la clé publique peut être distribuée ouvertement, tandis que la clé privée reste secrète. \\
%         \midrule
%         Usage typique           & Favorisé pour le chiffrement des données en transit, comme dans le cas des cellules Tor, où la performance et l'efficacité sont cruciales.        & Utilisé pour les échanges sécurisés initiaux, tels que l'établissement de clés symétriques ou la vérification d'identité, grâce à la sécurité renforcée qu'il offre. \\
%         \bottomrule
%     \end{tabularx}
%     \caption{Comparaison entre clés symétriques et asymétriques}
%     \label{tab:cles-sym-asym}
% \end{table}


% \begin{table}[htbp]
%     \centering
%     \begin{tabularx}{\textwidth}{
%         >{\raggedright\arraybackslash}p{2cm}
%         >{\raggedright\arraybackslash}X
%         >{\raggedright\arraybackslash}X}
%         \toprule
%         \rowcolor[HTML]{EFEFEF}
%         \textbf{}                   & \textbf{Cryptographie Symétrique} & \textbf{Cryptographie Asymétrique} \\
%         \midrule
%         Méthode de chiffrement      & Utilise une seule clé pour chiffrer et déchiffrer les données. & Utilise une paire de clés: une publique pour chiffrer, une privée pour déchiffrer. \\
%         \midrule
%         Algorithmes                 & AES et DES &  \acrshort{rsa} et \acrshort{ecc} \\
%         \midrule
%         Complexité algorithmique    & Opérations basées principalement sur des transformations linéaires simples (par exemple, permutation et substitution). Moins coûteux en termes de ressources de calcul. & Nécessite des opérations mathématiques complexes telles que l'exponentiation modulaire, ce qui entraîne des temps de traitement plus longs et une consommation plus élevée de ressources. \\
%         \midrule
%         Gestion des clés            & La distribution sécurisée des clés pose un défi majeur car chaque paire d'utilisateurs nécessite une clé unique partagée, escaladant le nombre de clés nécessaire de manière exponentielle avec le nombre d'utilisateurs. & Facilite la distribution des clés; seule la clé publique doit être partagée ouvertement, tandis que la clé privée est gardée secrète par chaque utilisateur, simplifiant la gestion des clés même dans de grands réseaux. \\
%         \midrule
%         Utilisation                 & Idéal pour les environnements où la sécurité des lignes de communication des clés peut être garantie. Couramment utilisé pour le chiffrement de données au repos et le chiffrement de masse de données en transit. & Utilisé principalement pour les transactions sécurisées, l'authentification et les signatures numériques où les clés ne doivent pas être échangées via un canal sécurisé, réduisant ainsi le risque d'exposition des clés. \\
%         \midrule
%         Vulnérabilités typiques     & Susceptible à l'analyse des clés si des clés insuffisamment aléatoires ou courtes sont utilisées, ou si la gestion des clés est compromise. & Plus vulnérable aux attaques cryptanalytiques telles que l'attaque par facteurisation pour \acrshort{rsa} ou les attaques sur la logique elliptique pour ECC, en particulier si des paramètres faibles ou mal configurés sont utilisés. \\
%         \bottomrule
%     \end{tabularx}
%     \caption{Comparaison entre la cryptographie symétrique et asymétrique}
%     \label{tab:tech-crypto}
% \end{table}




\begin{table}[htbp]
    \centering
    \begin{tabularx}{\textwidth}{
        >{\raggedright\arraybackslash}p{2cm} 
        >{\raggedright\arraybackslash}X 
        >{\raggedright\arraybackslash}X}
        \toprule
        \rowcolor[HTML]{EFEFEF}
        \textbf{Caractéristique} & \textbf{Cryptographie Symétrique} & \textbf{Cryptographie Asymétrique} \\
        \midrule
        Clés & Unique, partagée & Paire de clés (publique, privée) \\
        \midrule
        Algorithmes & AES, DES & RSA, \acrshort{ecc} \\
        \midrule
        Complexité & Moindre & Élevée \\
        \midrule
        Gestion des clés & Difficile pour de nombreux utilisateurs & Plus simple grâce à la clé publique \\
        \midrule
        Utilisation & Chiffrement de masse, sécurisé si la clé est sûre & Transactions sécurisées, signatures numériques \\
        \midrule
        Vulnérabilités & Analyse des clés, gestion compromise & Attaques cryptanalytiques, paramètres faibles \\
        \bottomrule
    \end{tabularx}
    \caption{Comparaison de la cryptographie symétrique et asymétrique}
    \label{tab:tech-crypto}
\end{table}
    
    