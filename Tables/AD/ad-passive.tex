\begin{table}[htpb]
    \centering
    \begin{tabularx}{\textwidth}{
        >{\raggedright\arraybackslash}X
        % p{4.5cm}
        >{\raggedright\arraybackslash}X}
    \toprule
    \rowcolor[HTML]{EFEFEF}
    \textbf{Menace}                          & \textbf{Solution} \\ 
    \midrule
    Observation des profils de trafic         & Le multiplexage de traffics sur un même circuit empêche l'analyse précise de profils de trafics. \\
    \midrule
    Observation du contenu utilisateur       & Si la destination est un noeud hostile, l'usage de \gls{privoxy}\footnote{Un privoxy est un proxy logiciel conçu spécialement pour \acrshort{tor} qui nettoye les protocoles d'application.} permet de conserver l'anonymat du client. \\
    \midrule
    Distinguabilité des options               & En permettant aux utilisateurs de confugurer leurs profils, cela compromet leur anonymat. \\
    \midrule
    Corrélation temporelle de bout en bout   & Masquer la communication entre le \acrshort{op} et le premier \acrshort{or} via un \acrshort{op} sur le \acrshort{or}. Cela contraint à séparer le traffic en provenance du trafic traversant. \\
    \midrule
    Corrélation de taille de bout en bout    & Rembourage\footnote{nombre différent de paquets autorisés} ou topologie du tuyau fuyant\footnote{Le tuyau fuyant (leaky pipe) est une topologie de circuit utilisée dans le réseau Tor pour améliorer l'anonymat des utilisateurs. Dans cette configuration, le trafic peut être dirigé pour sortir du circuit à des nœuds intermédiaires plutôt qu'à la fin du circuit. }. \\
    \midrule
    Empreintes digitales de site Web         & Multiplexage des flux, modification de la taille des cellules et rembourages. \\
    \bottomrule
    \end{tabularx}
    \caption{Menaces et défenses dans \acrshort{tor} : attaques passives}
    \label{tab:ad-passive}
\end{table}