\begin{table}[htpb]
\centering
\begin{tabularx}{\textwidth}{
    >{\raggedright\arraybackslash}p{4cm}
    >{\raggedright\arraybackslash}p{3cm}
    >{\raggedright\arraybackslash}X}
\toprule
\textbf{Catégorie} & \textbf{Elément} & \textbf{Description} \\
\midrule
\textbf{Liste des Routeurs d'Oignons} & 
   \textbf{Identifiants} & Chaque routeur a une clé publique unique utilisée pour l'identifier. \\
   & \textbf{Adresses IP et Ports} & Les adresses réseau et les ports sur lesquels les routeurs écoutent. \\
   & \textbf{Politiques de Sortie} & Les types de trafic que les routeurs acceptent ou rejettent (par exemple, blocage de certains ports). \\
\midrule
\textbf{État des Routeurs} &
    \textbf{Disponibilité} & Indique si un routeur est actuellement en ligne et accessible. \\
    & \textbf{Capacités} & Bande passante disponible, latence, et autres métriques de performance.\\
    & \textbf{Charges de Travail} & Informations sur la charge actuelle et la capacité du routeur à gérer plus de trafic. \\
 & \\
\midrule
\textbf{Clés Cryptographiques} &
    \textbf{Clés d'Identité} & Clés publiques utilisées pour vérifier l'identité du routeur. \\
    & \textbf{Certificats} & Signatures numériques qui certifient l'authenticité des clés et des informations publiées. \\
 & \\
\midrule
\textbf{Descripteurs de Serveurs} &
    \textbf{Descriptions Signées} & Déclarations signées par les routeurs eux-mêmes, détaillant leur état et leurs capacités. \\
    & \textbf{Horodatages} & Indiquent le moment où les informations ont été mises à jour pour assurer leur fraîcheur. \\
 & \\
\midrule
\textbf{Informations de Réseau} &
    \textbf{Topologie} & Vue d'ensemble de la disposition et des connexions entre les routeurs dans le réseau. \\
    & \textbf{Consensus} & Ensemble de données consolidées approuvé par plusieurs serveurs d'annuaire pour garantir la cohérence et la fiabilité des informations. \\
 & \\
\bottomrule
\end{tabularx}
\caption{Structure et contenu d'un annuaire.}
\label{tab:structure-annuaire}
\end{table}