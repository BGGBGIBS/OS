\begin{table}[htpb]
    \centering
    \begin{tabularx}{\textwidth}{
        >{\raggedright\arraybackslash}p{4.5cm}
        >{\raggedright\arraybackslash}X}
    \toprule
    \rowcolor[HTML]{EFEFEF}
    \textbf{Menace}                                         & \textbf{Solution} \\
    \midrule
    Compromission des clés                                  & La rotation periodique permet de contrer la compromission des clés de sessions \acrshort{tls} car même si l'attaquant peut observer les cellules relayées sur chacun des circuits de cette connexion, sans la clé oignon, il ne peut déchiffrer le contenu. \\ 
    \midrule
    Compromission itérative                                 & Le secret parfait vers l'avant (\textit{i.e.} Perfect Forward Secrecy) empêche de remonter un circuit même si un attaque a lieu d'un \acrshort{or} intermédiaire jusqu'au dernier. \\ 
    \midrule
    Exploitation d'un serveur web                           & \acrshort{tor} dépend de \gls{privoxy} pour empêcher un serveur web d'identifier des profils temporels des utilisateurs qui s'y connectent d'adapter ses réponses. \\
    \midrule
    Exploitation d'un proxy onion                           & À un client ne peut correspondre qu'un \acrshort{op} local. Cependant, compromettre un proxy onion compromet toutes ses connexions. \\ 
    \midrule
    Attaque DoS sur des nœuds non observables               & La mise en place de stratégies permettant au réseau de garder le contrôle sur les ressources matérielles utilisables permet de contrer les attaques de type Déni de service (\textit{i.e.} \acrfull{dos}). \\
    \midrule
    Exploitation d'un \acrshort{or} hostile                 & Un nœud hostile doit etre immediatement adjacent aux deux extremites pour compromettre l'anonymat d'un circuit. Si un adversaire contrôle plus d'un \acrshort{or}, il ne peut compromettre plus qu'une partie du trafic. \\ 
    \midrule
    Introduction de timing dans les messages                & Cela représente une version plus forte des attaques de timing passives déjà discutées. \\ 
    \midrule
    Attaques par marquage                                   & Les contrôles d'intégrité\footnote{Initialisation d'un digest SHA-1} sur les cellules empêchent un n\oe ud hostile de marquer une cellule en la modifiant. \\ 
    \midrule
    Remplacement de contenus de protocoles non authentifiés & La préférence des clients pour des protocoles authentifiés de bout en bout permet d'empêcher un nœud de sortie hostile de se faire passer pour le serveur cible. \\
    \midrule
    Attaques par rejeu                                      & Le protocole d'échnage de clés \acrshort{dh} permet de renégocier de nouvelles clés de session et ainsi contrer les attaques par replay. \\ 
    \midrule
    Attaques de diffamation                                 & Les politiques de sorties réduisent l'impact des attaques par diffamation. \\ 
    \midrule
    Distribution de code hostile                            & Des clés publiques de version officielles du code de \acrshort{tor} permettent de vérifier son authenticité avant de l'exécuter. \\ 
    \bottomrule
    \end{tabularx}
    \caption{Menaces et défenses dans \acrshort{tor} : attaques actives}
    \label{tab:ad-active}
\end{table}
