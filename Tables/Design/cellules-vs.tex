\begin{table}[htbp]
    \centering
    \begin{tabularx}{\textwidth}{
        >{\raggedright\arraybackslash}p{2cm} 
        >{\raggedright\arraybackslash}X 
        >{\raggedright\arraybackslash}X}
            \toprule
            \rowcolor[HTML]{EFEFEF}
            \textbf{Type}           & \textbf{Commandes}        & \textbf{Fonctions} \\
            \midrule
            Contrôle & PADDING & Garder la connexion active \\
            & CREATE & Établir un nouveau circuit \\
            & CREATED & Confirmer la création du circuit \\
            & DESTROY & Fermer un circuit \\
            \addlinespace
            Relais & RELAY\_BEGIN & démarrer une connexion \acrshort{tcp} \\
            & RELAY\_DATA & transfert de données \\
            & RELAY\_END & fermer une connexion \acrshort{tcp} \\
            & RELAY\_CONNECTED & confirmer l'établissement de la connexion \acrshort{tcp} \\
            & RELAY\_EXTEND & étendre un circuit à un autre nœud \\
            & RELAY\_TRUNCATED & signaler une coupure de circuit partielle \\
            & RELAY\_SENDME & contrôle de congestion (demande de données) \\
            \bottomrule
    \end{tabularx}
    \caption{Comparaison des Types de Cellules et de leurs commandes dans le Réseau Tor}
    \label{tab:cellules-tor}
\end{table}