\begin{table}[htbp]
    \centering
    \begin{tabularx}{\textwidth}{
        >{\raggedright\arraybackslash}p{2cm} 
        >{\raggedright\arraybackslash}X}
    \toprule
    \rowcolor[HTML]{EFEFEF}
    \textbf{Travail} & \textbf{Description} \\
    \midrule
    Chaum's Mix-Net & Chaum a proposé de cacher la correspondance entre l'expéditeur et le destinataire en enveloppant les messages dans des couches de cryptographie à clé publique et en les relayant à travers un chemin composé de "mélanges". Chaque mélange décrypte, retarde et réordonne les messages avant de les relayer. \\
    \midrule
    Babel, Mixmaster, Mixminion & Ces systèmes cherchent à maximiser l'anonymat au coût d'introduire des latences comparativement grandes et variables. Ils résistent aux adversaires globaux forts mais introduisent trop de décalage pour des tâches interactives comme la navigation web, le chat Internet ou les connexions SSH. \\
    \midrule
    \acrshort{tor} & \acrshort{tor} appartient à la catégorie des conceptions à faible latence qui tentent d'anonymiser le trafic réseau interactif. Ces systèmes gèrent une variété de protocoles bidirectionnels et fournissent une livraison de courrier plus pratique que les réseaux de courrier électronique anonymes à haute latence. Toutefois, ils peinent à empêcher un attaquant capable d'écouter les deux extrémités de la communication de corréler le timing et le volume du trafic entrant et sortant du réseau d'anonymat. \\
    \midrule
    Anonymizer (Single-hop proxies) & Ces conceptions utilisent un serveur unique de confiance pour supprimer l'origine des données avant de les relayer. Ces conceptions sont faciles à analyser mais les utilisateurs doivent faire confiance au proxy anonymisant. Concentrer le trafic en ce point unique augmente le jeu d'anonymat, mais cela devient vulnérable si l'adversaire peut observer tout le trafic entrant et sortant du proxy. \\
    \midrule
    Java Anon Proxy (JAP \acrshort{or} Web MIXes) & Utilise des routes partagées fixes appelées cascades. Comme avec un proxy à un seul saut, cette approche agrège les utilisateurs en ensembles d'anonymat plus larges, mais un attaquant n'a besoin d'observer que les deux extrémités de la cascade pour relier tout le trafic du système. \\
    \midrule
    PipeNet & Un autre design à faible latence, offrant une meilleure anonymat mais permettant à un seul utilisateur d'arrêter le réseau en ne transmettant pas de données. \\
    \midrule
    Tarzan, MorphMix, Crowds & Ces systèmes P2P tentent de dissimuler si un pair donné a initié une demande ou simplement relayé celle-ci. Tarzan et MorphMix utilisent le chiffrement par couches, tandis que Crowds suppose qu'un adversaire ne peut pas observer l'initiateur. Hordes, basé sur Crowds, utilise des réponses multicast pour masquer l'initiateur. \\
    \midrule
    Freedom, original Onion Routing & Construisent des circuits d'un seul coup en utilisant des couches de messages chiffrés avec des clés publiques. Tor, Tarzan, MorphMix, Cebolla, et le réseau d'anonymat de Rennhard construisent des circuits par étapes, en les étendant un saut à la fois. \\
    \bottomrule
    \end{tabularx}
    \caption{Systèmes d'anonymat}
    \label{tab:systems}
    \end{table}
    