contexte 
l'os c'est la base
northbridge = controler mémoire
le gestionnaire de taches permet de savoir quels processus sont en cours d'utilisation et quels compoants physiquer leur alloue des ressources



concepts de tout système d'exploitation
vérifier l'existence des fichiers 
les ressources sont utilisées correctement

typologies d'os
en terme de plus à moins puissante

exokernel : le logiciel décide ce qu'il utilise dans les ressources phyqieu du sytème

Processeur : mot binaire de la valeure la plus longue qu'un prcesseur puisse prendre en charge
10^3 = 1000
2^10 = 1024 = KibiByte
lw load lire le contenu d'un registre
sw = écrire le contenu d'un registre
le processeur décode et exécute les intructions stockées dans la mémoire

le compilateur va traduire le code écrit en language haut niveau en code binaure pour le cpu qui va s'appyer sur la rame


En assembleur, le fait de factoriser le code consiste à rendre un bout de code réutilisale via appel de fonction
