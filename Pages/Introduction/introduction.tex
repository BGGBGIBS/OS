\section{Introduction}\label{sec:introduction}

\lettrine{À}{l'aube} du XXIe siècle, jamais les aspects liés à la sécurité de nos interactions en ligne n'auront été aussi importants. 
Jamais nos moindres faits et gestes n'auront été autant traqués et analysés. 
Jamais nos données personnelles n'auront été aussi précieuses. 
Jamais nos vies privées n'auront été aussi publiques. 
Jamais notre sécurité en ligne n'aura autant impacté notre sécurité hors ligne.

\lettrine{Q}{ue ce soit} lors d'un virement en ligne, lors de téléchargements de contenus ou encore lors de recherches, tous nos usages se voient traqués et analysés. 
Tous sans exception voient nos données personnelles s'envoler à jamais dans les méandres du réseau internet mondial. 
Tous sans exception voient nos vies privées rendues publiques à l'instant où nous nous connectons.
Si ces inconforts mettent en lumière les failles des services dont on ne sait plus se passer, ils ne mettent cependant pas nos vie en danger, ni ne restreignent notre accès à l'information. 
Cela est dû au fait que nous nous connectons depuis les lieux les plus sûrs au monde, hors ligne.

\lettrine{Q}{'en} est-il des citoyens de régimes répressifs ? 
Des journalistes opérant depuis des zones à risque ? 
Des personnes devant transmettre des informations sensibles via internet ? 
C'est bien ici, pour ces usages particuliers, que l'existence même de \acrshort{tor} se justifie. 
Pour le citoyen Russe souhaitant accéder à des sources d'informations censurées, pour le journaliste opérant depuis une zone sous conflit armé, pour le lanceur d'alerte ne souhaitant pas être inquiété, pour les besoins militaires nécessitant des transferts de données fiables et intraçables. 
C'est bien dans ces contextes si particuliers que le routage onion de seconde génération ainsi que les services qu'il propose se révèlent cruciaux.

\lettrine{L}{e protocole} \acrfull{tor} est un réseau informatique décentralisé permettant l'anonymat, l'intégrité et la non vulnérabilité des données transmises et rendant le traçage sur le réseau internet mondial complexe.
En permettant une navigation anonyme et en protégeant les données contre l'analyse de trafic, \acrshort{tor} joue un rôle crucial dans la préservation tant de la vie privée en ligne que de la non-vulnérabilité des données transmises.
Et ce, tant pour des applications civiles que pour des usages plus sensibles liés à la sécurité nationale et à la protection des sources journalistiques.

\lettrine{C}{e travail} se propose d'explorer en profondeur le protocole \acrshort{tor}, en mettant en lumière non seulement son architecture de routage en oignon de deuxième génération mais aussi son évolution depuis sa création. 
À travers une analyse rigoureuse de l'article fondateur "TOR: The Second-Generation Onion Router" \cite[Tor]{dingledine_tor_2004} de Dingledine, Mathewson, et Syverson, ce rapport aspire à offir une compréhension nuancée des mécanismes qui sous-tendent le fonctionnement de \acrshort{tor} ainsi que les défis sécuritaires qu'il cherche à surmonter, contribuant ainsi à une meilleure appréciation de son impact sur les paradigmes de sécurité et d'anonymat sur Internet.

\lettrine{O}{utre} cette introduction, la section \ref{sec:description} "\nameref{sec:description}" présentera une vue d'ensemble de \acrshort{tor}, posant le contexte.
La section \ref{sec:works} "\nameref{sec:works}" décrira le fonctionnement du protocole de manière détaillée.
La section \ref{sec:security} "\nameref{sec:security}" examinera les aspects sécuritaires liés à \acrshort{tor}, tandis que la section \ref{sec:v3} "\nameref{sec:v3}" envisagera les perspectives d'évolution du protocole. 
Enfin, la section \ref{sec:conclusion} "\nameref{sec:conclusion}" conclura en résumant les principales trouvailles et recommandations découlant de cette analyse.
