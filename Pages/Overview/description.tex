\section{Description}\label{sec:description}

L'anonymat en ligne, objet de ce rapport au travers de l'analyse de \acrshort{tor}, va de pair avec une latence accrue en raison des différentes techniques et méthodes mises en place pour y parvenir.
Si \acrshort{tor} et le routage oignon sur lequel il repose font partie des systèmes dits de faible latence, les premiers à offrir l'anonymat appartenaient à une seconde catégorie dite à forte latence.

\textbf{Prémices de l'anonymat (0G)}

En 1981, Chaum introduisit la notion de mixnets (réseaux mélangés) \cite{chaum_cmix_nodate} afin de permettre aux utilisateurs d'utiliser un réseau sans compromettre leur anonymat.
Il était alors question de redéfinir l'ordre dans lequel les messages étaient transmis à travers le réseau: un message pouvait arriver en premier dans un noeud, mais être transmis après les 3 messages suivants.
Ce système permettait d'empêcher le suivi des données entre noeuds.
Si l'anonymat était garanti, des taux de latence particulièrement élevés venaient restreindre son utilité pour des usages nécessitant des temps de réponses réduits.
En effet, les mixnets étants basés uniquement sur la \acrfull{ac}, ils ne sont pas adaptés à des usages interactifs.

\textbf{Routage en Oignon (1G)}

Initialement développé dans les années 1990, le routage en oignon est venu résoudre ces problèmes en anonymisant des applications basées sur TCP, telles que la navigation web et la messagerie instantanée et les connexions SSH avec un taux de latence réduit.
Ce système, bien que novateur, présentait plusieurs lacunes significatives, notamment l'absence de sécurité parfaite vers l'avant, qui signifie que la sécurité des données n'était pas garantie si une partie du chemin était compromise après la transmission des données. 
De plus, il nécessitait des proxies d'application distincts pour chaque protocole de la 7e couche du modèle OSI supporté (\textit{e.g.,} HTTPS, FTP,\dots).
Ce qui a limité sa polyvalence ainsi que son efficacité face à divers types de traffics en ajoutant des couches de traitements supplémentaires.


% \textbf{Alternatives}

% \begin{table}[htbp]
    \centering
    \begin{tabularx}{\textwidth}{
        >{\raggedright\arraybackslash}p{2cm} 
        >{\raggedright\arraybackslash}X}
    \toprule
    \rowcolor[HTML]{EFEFEF}
    \textbf{Travail} & \textbf{Description} \\
    \midrule
    Chaum's Mix-Net & Chaum a proposé de cacher la correspondance entre l'expéditeur et le destinataire en enveloppant les messages dans des couches de cryptographie à clé publique et en les relayant à travers un chemin composé de "mélanges". Chaque mélange décrypte, retarde et réordonne les messages avant de les relayer. \\
    \midrule
    Babel, Mixmaster, Mixminion & Ces systèmes cherchent à maximiser l'anonymat au coût d'introduire des latences comparativement grandes et variables. Ils résistent aux adversaires globaux forts mais introduisent trop de décalage pour des tâches interactives comme la navigation web, le chat Internet ou les connexions SSH. \\
    \midrule
    \acrshort{tor} & \acrshort{tor} appartient à la catégorie des conceptions à faible latence qui tentent d'anonymiser le trafic réseau interactif. Ces systèmes gèrent une variété de protocoles bidirectionnels et fournissent une livraison de courrier plus pratique que les réseaux de courrier électronique anonymes à haute latence. Toutefois, ils peinent à empêcher un attaquant capable d'écouter les deux extrémités de la communication de corréler le timing et le volume du trafic entrant et sortant du réseau d'anonymat. \\
    \midrule
    Anonymizer (Single-hop proxies) & Ces conceptions utilisent un serveur unique de confiance pour supprimer l'origine des données avant de les relayer. Ces conceptions sont faciles à analyser mais les utilisateurs doivent faire confiance au proxy anonymisant. Concentrer le trafic en ce point unique augmente le jeu d'anonymat, mais cela devient vulnérable si l'adversaire peut observer tout le trafic entrant et sortant du proxy. \\
    \midrule
    Java Anon Proxy (JAP \acrshort{or} Web MIXes) & Utilise des routes partagées fixes appelées cascades. Comme avec un proxy à un seul saut, cette approche agrège les utilisateurs en ensembles d'anonymat plus larges, mais un attaquant n'a besoin d'observer que les deux extrémités de la cascade pour relier tout le trafic du système. \\
    \midrule
    PipeNet & Un autre design à faible latence, offrant une meilleure anonymat mais permettant à un seul utilisateur d'arrêter le réseau en ne transmettant pas de données. \\
    \midrule
    Tarzan, MorphMix, Crowds & Ces systèmes P2P tentent de dissimuler si un pair donné a initié une demande ou simplement relayé celle-ci. Tarzan et MorphMix utilisent le chiffrement par couches, tandis que Crowds suppose qu'un adversaire ne peut pas observer l'initiateur. Hordes, basé sur Crowds, utilise des réponses multicast pour masquer l'initiateur. \\
    \midrule
    Freedom, original Onion Routing & Construisent des circuits d'un seul coup en utilisant des couches de messages chiffrés avec des clés publiques. Tor, Tarzan, MorphMix, Cebolla, et le réseau d'anonymat de Rennhard construisent des circuits par étapes, en les étendant un saut à la fois. \\
    \bottomrule
    \end{tabularx}
    \caption{Systèmes d'anonymat}
    \label{tab:systems}
    \end{table}
    
% \begin{table}[htbp]
    \centering
    \begin{tabularx}{\textwidth}{
        >{\raggedright\arraybackslash}X
        >{\raggedright\arraybackslash}X
        >{\raggedright\arraybackslash}X
        >{\raggedright\arraybackslash}X
        >{\raggedright\arraybackslash}X}
        \toprule
    \rowcolor[HTML]{EFEFEF}    
    \textbf{Système} & \textbf{Latence} & \textbf{Anonymat} & \textbf{Utilisation principale} & \textbf{Vulnérabilités} \\
    \midrule
    Chaum's Mix-Net & Haute & Très élevé & Messages asynchrones & Délais et réordonnancement \\
    \midrule
    Mixmaster & Haute & Très élevé & Email anonyme & Latence élevée, peu interactif \\
    \midrule
    Mixminion & Haute & Très élevé & Email anonyme & Latence élevée, peu interactif \\
    \midrule
    \acrshort{tor} & Faible & Élevé & Trafic interactif (Web, SSH) & Corrélation de timing et de volume \\
    \midrule
    Anonymizer & Très faible & Faible & Proxy unique, navigation Web & Confiance en un seul point, vulnérable à l'observation globale \\
    \midrule
    Java Anon Proxy (JAP) & Faible & Élevé & Web MIXes, routes fixes & Observation des extrémités de cascade \\
    \midrule
    PipeNet & Faible & Très élevé & Réseau à faible latence & Réseau arrêté par utilisateur unique \\
    \midrule
    Tarzan & Faible & Élevé & Réseau P2P & Complexité, difficulté de déploiement \\
    \midrule
    MorphMix & Faible & Élevé & Réseau P2P & Complexité, difficulté de déploiement \\
    \midrule
    Crowds & Faible & Moyen & Anonymat pour requêtes HTTP & Pas de chiffrement à clé publique \\
    \midrule
    Hordes & Faible & Moyen & Basé sur Crowds, réponses multicast & Latence, complexité accrue \\
    \midrule
    Freedom & Faible & Élevé & Réseau d'oignons original & Complexité, nécessités de patchs kernel \\
    \bottomrule
    \end{tabularx}
    \caption{Comparaison des systèmes d'anonymat}
    \label{tab:systems-vs}
    \end{table}
    

\textbf{Evolution vers \acrshort{tor} (2G)}

En réponse à ces défis, \acrfull{tor} a été introduit en 2002, marquant une évolution majeure du concept original. 

\begin{quote}
    \textit{Onion Routing is a distributed overlay network designed to anonymize TCP-based applications like web browsing, secure shell, and instant messaging.} \cite[p.~1, sec.~1]{dingledine_tor_2004}
\end{quote}

Il est important de noter que le protocole \acrshort{udp} n'est actuellement pas pris en charge par \acrshort{tor}.
En effet, ce dernier ne nécessitant pas de connexion et donc d'accusé de réception, cela le rend intrinsèquement plus complexe à anonymiser sans induire une latence importante qui serait contraire aux objectifs de conception du réseau.

\acrshort{tor} a introduit plusieurs améliorations significatives :
\begin{itemize}
    \item Sécurité parfaite vers l'avant : Cette fonctionnalité assure que, même si un nœud intermédiaire est compromis, les données transmises antérieurement restent protégées.
    \item Simplification des proxies d'application : Grâce à l'adoption de l'interface proxy standard SOCKS, \acrshort{tor} supporte plusieurs types de trafic \acrshort{tcp} sans nécessiter de modifications logicielles spécifiques, augmentant ainsi sa flexibilité et réduisant la complexité technique.
    \item Contrôle de congestion décentralisé : \acrshort{tor} améliore la réactivité du système et la gestion de la charge réseau par un mécanisme de contrôle de congestion qui ne nécessite pas de communication inter-nœuds, facilitant ainsi une meilleure scalabilité et performance du réseau.
\end{itemize}


Les utilisateurs établissent des circuits sécurisés à travers le réseau, où chaque nœud, ou "routeur oignon", ne connaît que le nœud précédent et le suivant. Les données transitent en cellules chiffrées, chaque nœud dévoilant progressivement le chemin jusqu'à la destination finale, ce qui préserve l'anonymat de la source et de la destination.

\textbf{Défis et Perspectives (3G)}

Malgré ses avancées, \acrshort{tor} fait face à des défis continus, notamment en matière de latence réseau et de résistance aux attaques d'analyse de trafic, où des adversaires sophistiqués pourraient théoriquement corréler les modèles de trafic entrant et sortant pour compromettre l'anonymat. Les efforts continus de la communauté pour mettre à jour et améliorer \acrshort{tor} sont cruciaux pour répondre aux défis de l'Internet moderne et maintenir la robustesse du système face aux menaces émergentes.

En résumé, \acrshort{tor} représente une solution robuste et flexible pour l'anonymat en ligne, mais comme tout système, il n'est pas exempt de limitations qui doivent être adressées pour assurer son efficacité à long terme.
La Section \ref{sec:v3} "\nameref{sec:v3}" concerne la 3e génération du routage oignon.