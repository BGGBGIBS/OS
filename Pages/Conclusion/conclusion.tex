\section{Conclusion}\label{sec:conclusion}

Tout au long du présent rapport, nous avons pu nous rendre compte des avancées considérables réalisées dans \acrfull{tor} en comparaison à l'onion routing de première génération.

De par son secret parfait vers l'avant, ses services cachés et autres options configurables, \acrshort{tor} se positionne aujourd'hui comme une référence avec une place de choix parmi les systèmes de communications permettant l'anonymat.

Cependant, cette seconde génération n'était pas exempte de défauts et a donné suite à une 3e version offrant un niveau de sécurité encore accru.

Pour palier aux défauts de \acrshort{tor}, d'autres solutions émergent dont TomiNet \cite{adeyanju_how_nodate}, produit de Tomi, une compagnie Web3. 
TomiNet entend résoudre 2 défauts de fabrication de \acrshort{tor}:
\begin{quote}
    \flqq \space When the internet was created, I am sure they wanted to allow freedom of information and speech, but because the architecture of the technology includes [internet protocols] IPs and centralized entities like [Internet Corporation for Assigned Names and Numbers] ICANN control domain names, the result is that governments can, through IPs, know who says what, go after them and block websites. \frqq{}, — Techno Prince, a pseudonymous member of Tomi’s founding team \parencite{adeyanju_how_nodate}
\end{quote} 

Si TomiNet repose sur une architecture similaire à celle de \acrshort{tor}, le réseau y ajoute une couche de gouvernance décentralisée via \acrfull{dao} permettant ainsi de réduire les usages liés aux activités illicites. 