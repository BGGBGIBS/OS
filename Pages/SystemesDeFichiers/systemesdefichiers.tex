\section{Systèmes de fichiers}\label{sec:systemesdefichiers}

\subsection{Problématique}
La gestion de la mémoire secondaire est une tâche cruciale pour un système d'exploitation. 
Les systèmes de fichiers sont conçus pour organiser l'espace de stockage, permettant ainsi les opérations de lecture et d'écriture de données. 
Les objectifs principaux sont d'assurer la pérennité des données au-delà de l'exécution des processus et de protéger l'accès aux fichiers via des permissions.

\subsection{Arborescence des fichiers}
Le contenu d'un espace de stockage est structuré sous forme d'arborescence, où les nœuds peuvent être des fichiers ou des répertoires. 
Chaque fichier est identifié par un chemin. 
Les répertoires regroupent des fichiers et/ou d'autres répertoires, facilitant ainsi la navigation et l'organisation des données.

\subsection{Supports physiques}
Les supports physiques de stockage varient en termes de caractéristiques et d'usages :
\begin{itemize}
    \item \textbf{HDD} : Offre un grand espace de stockage à faible coût avec des temps d'accès variables.
    \item \textbf{SSD} : Plus coûteux mais avec des temps d'accès beaucoup plus rapides et constants.
    \item \textbf{Bandes magnétiques} : Utilisées principalement pour les sauvegardes en raison de leur grande fiabilité, mais adaptées uniquement pour des accès séquentiels.
\end{itemize}

\subsection{Modes d'allocation}
Il existe plusieurs stratégies d'allocation pour les blocs de mémoire :
\begin{itemize}
    \item \textbf{Allocation contiguë} : Les fichiers sont stockés dans des blocs contigus, ce qui facilite la lecture séquentielle mais peut causer de la fragmentation externe et des difficultés d'extension.
    \item \textbf{Allocation chaînée} : Utilise des pointeurs pour lier les blocs de données, réduisant ainsi la fragmentation externe mais augmentant le risque de perte de données en cas de corruption.
    \item \textbf{Allocation par i-nodes} : Utilise une structure de données appelée i-node pour garder la trace des blocs de fichiers, supportant des accès séquentiels et directs tout en nécessitant un accès supplémentaire pour les i-nodes indirects.
\end{itemize}

\subsection{Performances et cohérence}
Pour améliorer les performances, des techniques comme l'utilisation d'une cache de blocs (pour stocker les blocs fréquemment accédés) et la journalisation (pour assurer la cohérence des écritures) sont employées. 
La journalisation consiste à enregistrer les opérations avant de les appliquer et à les supprimer une fois complétées, minimisant ainsi les risques de corruption en cas de panne.

\subsection{Illustrations}
\textbf{FAT16} : Dans un système de fichiers FAT16, les clusters sont des unités d'allocation de taille fixe. 
La table d'allocation de fichiers (FAT) garde la trace des clusters alloués à chaque fichier. 
Ce système permet de gérer la fragmentation externe et d'optimiser l'utilisation de l'espace de stockage disponible.

\subsection{Conclusion}
Les systèmes de fichiers jouent un rôle essentiel dans la gestion des données, offrant des solutions variées pour répondre aux besoins de pérennité, de protection et de performance. 
La compréhension des différents modes d'allocation et des caractéristiques des supports physiques est cruciale pour optimiser leur utilisation dans un environnement de système d'exploitation.
