\section{Processus}\label{sec:processus}

\subsection{Problématique}
Un processus est une instance d'un programme en cours d'exécution, comprenant son code, ses données et ses ressources système allouées. 
La gestion des processus est essentielle pour assurer l'exécution efficace et sécurisée des programmes sur un système informatique.

\subsection{Constitution d'un processus}
Un processus se compose de plusieurs éléments :
\begin{itemize}
    \item \textbf{Espace d'adressage} : Comprend le code du programme, les données statiques, la pile et le tas.
    \item \textbf{Descripteur de processus} : Contient l'identifiant unique (PID), l'état, les registres, le pointeur de pile, et les descripteurs de fichiers ouverts.
\end{itemize}

\subsection{Cycle de vie d'un processus}
Le cycle de vie d'un processus inclut plusieurs états :
\begin{itemize}
    \item \textbf{Création} : Un processus parent crée un processus enfant via un appel système tel que fork().
    \item \textbf{Exécution} : Le processus utilise le processeur pour exécuter les instructions.
    \item \textbf{Prêt} : Le processus est en attente de l'allocation du processeur.
    \item \textbf{Bloqué} : Le processus attend qu'une condition soit remplie (e.g., fin d'I/O).
    \item \textbf{Terminaison} : Le processus libère ses ressources et passe à l'état zombi jusqu'à ce que son descripteur soit nettoyé par le processus parent.
\end{itemize}

\subsection{Ordonnancement des processus}
L'ordonnancement est la méthode utilisée par le système d'exploitation pour attribuer du temps processeur aux processus prêts à s'exécuter. 
Les principales politiques d'ordonnancement incluent :
\begin{itemize}
    \item \textbf{First-Come, First-Served (FCFS)} : Les processus sont exécutés dans l'ordre de leur arrivée.
    \item \textbf{Round-Robin} : Chaque processus reçoit un quantum de temps fixe pour s'exécuter tour à tour.
    \item \textbf{Shortest Job Next (SJN)} : Le processus avec le temps d'exécution le plus court est exécuté en premier.
\end{itemize}

\subsubsection{Systèmes batch}

Les systèmes batch exécutent des lots de travaux sans interaction avec avec l'utilisateur pendant l'exécution.

Algorithmes
\begin{itemize}
    \item First-Come, First-Served (FCFS): Les processus sont exécutés dans l'ordre de leur arrivée.
    \item Shortest Job Next (SJN): Les processus avec le plus court temps d'exécution sont prioritaires
\end{itemize}
\subsubsection{Systèmes interactifs}

Les systèmes interactifs permettent une interaction continue avec l'utilisateur, nécessitant des temps de réponse courts.

Algorithmes:
\begin{itemize}
    \item Round Robin: chaque processus reçoit un temps de CPU fixe, appelé quantum, et est ensuite placé à la fin de la file d'attente.
    \item Priority scheduling: les processus sont exécutés en fonction de leur priorité.
\end{itemize}
\subsubsection{Systèmes en temps réel}

Les systèmes en temps réels doivent répondre à des contraintes temporelles strictes, souvent utilisés dans des applications critiques.

Algorithmes:
\begin{itemize}
    \item Rate Monotonic Scheduling (RMS): les processus avec les périodes les plus courtes ont la plus haute priorité
    \item Earliest Deadline First (EDF) : Les processus sont exécutés en fonction de leur échéance la plus proche
\end{itemize}

\subsection{Multi-threading}
Le multi-threading permet à un processus de contenir plusieurs threads d'exécution, ce qui améliore l'efficacité et la réactivité. Les threads partagent l'espace d'adressage du processus mais possèdent des contextes d'exécution indépendants.

\subsection{Gestion des événements et des interruptions}
Les interruptions permettent de gérer des événements asynchrones (e.g., I/O). 
Lorsqu'une interruption se produit, le système d'exploitation sauvegarde le contexte du processus courant et exécute le gestionnaire d'interruption approprié.

\subsection{Conclusion}
La gestion des processus est fondamentale pour le fonctionnement d'un système d'exploitation, impactant directement l'efficacité et la stabilité des systèmes informatiques. 
Une bonne compréhension de la constitution des processus, de leur cycle de vie, des stratégies d'ordonnancement et du multi-threading est essentielle pour développer des applications performantes et robustes.
