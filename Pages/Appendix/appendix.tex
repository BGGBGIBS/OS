\appendix
% \section{Différence entre \acrshort{tls} et le chiffrement par clé}
% \begin{table}[htbp]
    \centering
    \begin{tabularx}{\textwidth}{
        >{\raggedright\arraybackslash}p{2cm} 
        >{\raggedright\arraybackslash}X 
        >{\raggedright\arraybackslash}X}
        \toprule
        \rowcolor[HTML]{EFEFEF}
        \textbf{}           & \textbf{TLS}                                                                      & \textbf{Chiffrement par clé} \\
        \midrule
        Nature              & Protocole de sécurité pour les communications réseau                              & Méthode pour chiffrer/déchiffrer des données \\
        \midrule
        Utilisation de clés & Utilise à la fois des clés symétriques et asymétriques                            & Peut utiliser des clés symétriques ou asymétriques \\
        \midrule
        Objectif principal  & Sécuriser des sessions ou connexions entières                                     & Sécuriser des données ou messages spécifiques \\
        \midrule
        Mise en œuvre       & Négocie les paramètres de chiffrement avant le transfert de données               & Appliqué aux données elles-mêmes, indépendamment du canal \\
        \midrule
        Applications        & Sécurisation du web (HTTPS), courrier électronique, messagerie instantanée, etc.  & Chiffrement de fichiers, communication sécurisée, authentification, signature numérique \\
        \bottomrule
    \end{tabularx}
    \caption{Comparaison entre \acrshort{tls} et le chiffrement par clé}
    \label{tab:TLSvsKeyEncryption}
\end{table}

% \section{Différents types d'attaques}
% \begin{longtable}{p{0.25\linewidth} p{0.7\linewidth}}
    \toprule
    \textbf{Attaque} & \textbf{Défense} \\
    \midrule
    Observation du trafic utilisateur & Le contenu à l'extrémité de l'utilisateur est crypté, bien que les connexions aux répondants puissent ne pas l'être. \\
    \addlinespace
    Corrélation de timing de bout en bout & \acrshort{tor} minimise mais ne cache pas complètement ces corrélations. \\
    \addlinespace
    Corrélation de la taille de bout en bout & La multiplexation des flux dans le même circuit limite cette corrélation à la granularité des cellules. \\
    \addlinespace
    Empreinte digitale de site web & Potentiellement effective contre Tor; des défenses supplémentaires pourraient inclure des stratégies de padding. \\
    \addlinespace
    Compromission de clés & La rotation périodique des clés limite la fenêtre d'opportunité pour ces attaques. \\
    \addlinespace
    Compromission d'un \acrshort{or} par itération & \acrshort{tor} protège contre ce scénario en éliminant rapidement les informations nécessaires pour compléter une telle attaque. \\
    \addlinespace
    Exécution d'un serveur de destination hostile & Les attaques de fin à fin deviennent plus faciles si l'adversaire peut induire les utilisateurs à se connecter à son serveur web. \\
    \addlinespace
    Exécution d'un \acrshort{or} hostile & Un \acrshort{or} isolé hostile peut créer des circuits à travers lui-même ou altérer des modèles de trafic. \\
    \addlinespace
    Introduire un timing dans les messages & \acrshort{tor} n'adresse pas directement ces attaques, mais elles ne sont qu'une version plus forte des attaques passives de timing. \\
    \addlinespace
    Attaques de tag & Les vérifications d'intégrité sur les cellules empêchent cette attaque. \\
    \addlinespace
    Remplacement de contenus de protocoles non authentifiés & Un nœud de sortie hostile peut usurper le serveur cible pour des protocoles non authentifiés. \\
    \addlinespace
    Attaques de replay & Non pertinentes pour Tor; la rediffusion d'un côté de la poignée de main entraîne une clé de session différente. \\
    \addlinespace
    Attaques contre les points de rendez-vous & Bob peut limiter la réception de requêtes via des tokens d'autorisation, et les points d'introduction peuvent être secrètement annoncés. \\
    \addlinespace
    Attaques contre les serveurs de répertoire & Si un serveur de répertoire est compromis, il peut partiellement influencer le répertoire final, mais une majorité est nécessaire pour inclure des \acrshort{or} compromis. \\
    \bottomrule
    \end{longtable}

\section{Latence et anonymat}
Dans les sytèmes de communication en réseau, le niveau d'anonymat est toujours corrélé avec le niveau de latence de ce système.

L'anonymat d'un client sur un réseau informatique lui confère l'intraçabilité de son activité en ligne.
Dans le cadre de \acrshort{tor}, cela est réalisé via le routage du trafic à travers plusieurs n\oe uds ou \acrfull{or}: chaque n\oe ud n'ayant connaissance que de son prédécesseur ainsi que de son successeur.

La latence - somme des délais - dans un réseau informatique désigne le temps écoulé entre l'envoi d'un paquet de données depuis une source et sa réception par la destination.
Dans le cadre de \acrshort{tor}, son accroissement s'explique d'une part par la répartition géographique globale des \acrshort{or} (\textit{cfr,} \acrfull{pd}) et d'autre part par les délais de chiffrement, déchiffrement qui surviennent à chaque \acrshort{or} (\textit{cfr,} \acrfull{td}).

Cette latence est donc un compromis nécessaire pour obtenir un niveau élevé d'anonymat et de sécurité, car elle rend plus difficile pour un observateur de corréler le trafic entrant et sortant.


\section{Menaces}\label{sec:ad}

Comme décrit tout au long de ce rapport, \acrshort{tor} assure l'anonymat de ses client, la non-vulnérabilité des données qu'ils transmettent ainsi que l'intraçabilité de leurs communications de bout en bout. 
La présente section s'attardera sur les différents types d'attaques recensées dans l'article ainsi que les axes de défenses mis en place dans \acrshort{tor}.
L'anonymat sera abordé au point \ref{subsubsec:passive} "\nameref{subsubsec:passive}", les aspects cryptographiques au point \ref{subsubsec:active} "\nameref{subsubsec:active}".
Les servrus de répertoires au point \ref{subsubsec:annuaire} "\nameref{subsubsec:annuaire}" et les point de rendez-vous au \ref{subsubsec:rdv} "\nameref{subsubsec:rdv}".

\subsection{Passive}\label{subsubsec:passive}

Les attaques passives concernent uniquement l'établissement de profils sur base d'analyses permettant ainsi de compromettre l'anonymat. 
Le tableau \ref{tab:ad-passive} "\nameref{tab:ad-passive}" ci-dessous présente les différents types d'attaques passives recensées dans l'article et y associe un axe de défense.
\begin{table}[htpb]
    \centering
    \begin{tabularx}{\textwidth}{
        >{\raggedright\arraybackslash}X
        % p{4.5cm}
        >{\raggedright\arraybackslash}X}
    \toprule
    \rowcolor[HTML]{EFEFEF}
    \textbf{Menace}                          & \textbf{Solution} \\ 
    \midrule
    Observation des profils de trafic         & Le multiplexage de traffics sur un même circuit empêche l'analyse précise de profils de trafics. \\
    \midrule
    Observation du contenu utilisateur       & Si la destination est un noeud hostile, l'usage de \gls{privoxy}\footnote{Un privoxy est un proxy logiciel conçu spécialement pour \acrshort{tor} qui nettoye les protocoles d'application.} permet de conserver l'anonymat du client. \\
    \midrule
    Distinguabilité des options               & En permettant aux utilisateurs de confugurer leurs profils, cela compromet leur anonymat. \\
    \midrule
    Corrélation temporelle de bout en bout   & Masquer la communication entre le \acrshort{op} et le premier \acrshort{or} via un \acrshort{op} sur le \acrshort{or}. Cela contraint à séparer le traffic en provenance du trafic traversant. \\
    \midrule
    Corrélation de taille de bout en bout    & Rembourage\footnote{nombre différent de paquets autorisés} ou topologie du tuyau fuyant\footnote{Le tuyau fuyant (leaky pipe) est une topologie de circuit utilisée dans le réseau Tor pour améliorer l'anonymat des utilisateurs. Dans cette configuration, le trafic peut être dirigé pour sortir du circuit à des nœuds intermédiaires plutôt qu'à la fin du circuit. }. \\
    \midrule
    Empreintes digitales de site Web         & Multiplexage des flux, modification de la taille des cellules et rembourages. \\
    \bottomrule
    \end{tabularx}
    \caption{Menaces et défenses dans \acrshort{tor} : attaques passives}
    \label{tab:ad-passive}
\end{table}

\newpage
\subsection{Active}\label{subsubsec:active}

Les attaqques actives concernent uniquement la compromission des techniques de chiffrement via divers procédés. 
Le tableau \ref{tab:ad-active} "\nameref{tab:ad-active}" ci-dessous présente les différents types d'attaques actives recencées dans l'article et y associe un axe de défense.
\begin{table}[htpb]
    \centering
    \begin{tabularx}{\textwidth}{
        >{\raggedright\arraybackslash}p{4.5cm}
        >{\raggedright\arraybackslash}X}
    \toprule
    \rowcolor[HTML]{EFEFEF}
    \textbf{Menace}                                         & \textbf{Solution} \\
    \midrule
    Compromission des clés                                  & La rotation periodique permet de contrer la compromission des clés de sessions \acrshort{tls} car même si l'attaquant peut observer les cellules relayées sur chacun des circuits de cette connexion, sans la clé oignon, il ne peut déchiffrer le contenu. \\ 
    \midrule
    Compromission itérative                                 & Le secret parfait vers l'avant (\textit{i.e.} Perfect Forward Secrecy) empêche de remonter un circuit même si un attaque a lieu d'un \acrshort{or} intermédiaire jusqu'au dernier. \\ 
    \midrule
    Exploitation d'un serveur web                           & \acrshort{tor} dépend de \gls{privoxy} pour empêcher un serveur web d'identifier des profils temporels des utilisateurs qui s'y connectent d'adapter ses réponses. \\
    \midrule
    Exploitation d'un proxy onion                           & À un client ne peut correspondre qu'un \acrshort{op} local. Cependant, compromettre un proxy onion compromet toutes ses connexions. \\ 
    \midrule
    Attaque DoS sur des nœuds non observables               & La mise en place de stratégies permettant au réseau de garder le contrôle sur les ressources matérielles utilisables permet de contrer les attaques de type Déni de service (\textit{i.e.} \acrfull{dos}). \\
    \midrule
    Exploitation d'un \acrshort{or} hostile                 & Un nœud hostile doit etre immediatement adjacent aux deux extremites pour compromettre l'anonymat d'un circuit. Si un adversaire contrôle plus d'un \acrshort{or}, il ne peut compromettre plus qu'une partie du trafic. \\ 
    \midrule
    Introduction de timing dans les messages                & Cela représente une version plus forte des attaques de timing passives déjà discutées. \\ 
    \midrule
    Attaques par marquage                                   & Les contrôles d'intégrité\footnote{Initialisation d'un digest SHA-1} sur les cellules empêchent un n\oe ud hostile de marquer une cellule en la modifiant. \\ 
    \midrule
    Remplacement de contenus de protocoles non authentifiés & La préférence des clients pour des protocoles authentifiés de bout en bout permet d'empêcher un nœud de sortie hostile de se faire passer pour le serveur cible. \\
    \midrule
    Attaques par rejeu                                      & Le protocole d'échnage de clés \acrshort{dh} permet de renégocier de nouvelles clés de session et ainsi contrer les attaques par replay. \\ 
    \midrule
    Attaques de diffamation                                 & Les politiques de sorties réduisent l'impact des attaques par diffamation. \\ 
    \midrule
    Distribution de code hostile                            & Des clés publiques de version officielles du code de \acrshort{tor} permettent de vérifier son authenticité avant de l'exécuter. \\ 
    \bottomrule
    \end{tabularx}
    \caption{Menaces et défenses dans \acrshort{tor} : attaques actives}
    \label{tab:ad-active}
\end{table}


\newpage
\subsection{Annuaire}\label{subsubsec:annuaire}
Les serveurs d'annuaires sont un coposante essentielle de tout réseau \acrshort{tor}.
% Le Tableau \ref{tab:structure-annuaire} "\nameref{tab:structure-annuaire}" ci-dessous reprend la structure ainsi que le contenu typique d'un annuaire.
% \begin{table}[htpb]
\centering
\begin{tabularx}{\textwidth}{
    >{\raggedright\arraybackslash}p{4cm}
    >{\raggedright\arraybackslash}p{3cm}
    >{\raggedright\arraybackslash}X}
\toprule
\textbf{Catégorie} & \textbf{Elément} & \textbf{Description} \\
\midrule
\textbf{Liste des Routeurs d'Oignons} & 
   \textbf{Identifiants} & Chaque routeur a une clé publique unique utilisée pour l'identifier. \\
   & \textbf{Adresses IP et Ports} & Les adresses réseau et les ports sur lesquels les routeurs écoutent. \\
   & \textbf{Politiques de Sortie} & Les types de trafic que les routeurs acceptent ou rejettent (par exemple, blocage de certains ports). \\
\midrule
\textbf{État des Routeurs} &
    \textbf{Disponibilité} & Indique si un routeur est actuellement en ligne et accessible. \\
    & \textbf{Capacités} & Bande passante disponible, latence, et autres métriques de performance.\\
    & \textbf{Charges de Travail} & Informations sur la charge actuelle et la capacité du routeur à gérer plus de trafic. \\
 & \\
\midrule
\textbf{Clés Cryptographiques} &
    \textbf{Clés d'Identité} & Clés publiques utilisées pour vérifier l'identité du routeur. \\
    & \textbf{Certificats} & Signatures numériques qui certifient l'authenticité des clés et des informations publiées. \\
 & \\
\midrule
\textbf{Descripteurs de Serveurs} &
    \textbf{Descriptions Signées} & Déclarations signées par les routeurs eux-mêmes, détaillant leur état et leurs capacités. \\
    & \textbf{Horodatages} & Indiquent le moment où les informations ont été mises à jour pour assurer leur fraîcheur. \\
 & \\
\midrule
\textbf{Informations de Réseau} &
    \textbf{Topologie} & Vue d'ensemble de la disposition et des connexions entre les routeurs dans le réseau. \\
    & \textbf{Consensus} & Ensemble de données consolidées approuvé par plusieurs serveurs d'annuaire pour garantir la cohérence et la fiabilité des informations. \\
 & \\
\bottomrule
\end{tabularx}
\caption{Structure et contenu d'un annuaire.}
\label{tab:structure-annuaire}
\end{table}

Les attaques contre les serveurs d'annuaires ciblent celles dont l'objectif est la compromission d'une partie des circuits créés.
Le tableau \ref{tab:ad-ds} "\nameref{tab:ad-rdvp}" ci-dessous présente les différents types d'attaques contre les serveurs d'annuaires et y associe un axe de défense.
\begin{table}[htpb]
    \centering
    \begin{tabularx}{\textwidth}{
        >{\raggedright\arraybackslash}p{4.5cm}
        >{\raggedright\arraybackslash}X}
    \toprule
    \rowcolor[HTML]{EFEFEF}
    \textbf{Menace}                                                         & \textbf{Solution} \\ 
    \midrule
    Destruction des serveurs d'annuaires                                    & Tant que la moitié des serveurs d'annuaires sont en activité, ils continuent de fournir un annuaire valide. \\
    \midrule
    Subversion d'un serveur d'annuaire                                      & N'a qu'une influence partielle et minime sur la composition de l'annuaire final. \\
    \midrule
    Subversion de la majorité des serveurs d'annuaire                       & Les opérateurs de serveurs d'annuaires doivent être indépendants et résistants. \\
    \midrule
    Encouragement à la dissension entre serveurs d'annuaire                 & Aucune solution n'est proposée pour empecher de diviser les opérateurs et ainsi leurs utilisateurs. \\
    \midrule
    Tromper les serveurs d'annuaire pour lister un \acrshort{or} hostile    & Les opérateurs de serveurs de annuaire sont capables de filtrer la plupart des \acrshort{or}s hostiles. \\
    \midrule
    Convaincre les annuaires qu'un \acrshort{or} défaillant fonctionne      & Les serveurs d'annuaires doivent impérativement tester les \acrshort{or} de manière appropriée pour empêcher qu'un \acrshort{or} ne puisse accepter une connexion \acrshort{tls} en ignorant les cellules et ainsi passer les barrières de sécurité de \acrshort{tor}. \\
    \bottomrule
    \end{tabularx}
    \caption{Menaces et défenses dans \acrshort{tor} : serveurs d'annuaires}
    \label{tab:ad-ds}
\end{table}


\subsection{Points de rendez-vous}\label{subsubsec:rdv}

Les attaques contre les points de rendez-vous.
Le tableau \ref{tab:ad-rdvp} "\nameref{tab:ad-rdvp}" ci-dessous recence les différent types d'attaques contre les points de rendez-vous et y associe un axe de défense.
\begin{table}[htpb]
    \centering
    \begin{tabularx}{\textwidth}{
        >{\raggedright\arraybackslash}p{4.5cm}
        >{\raggedright\arraybackslash}X}
            \toprule
            \rowcolor[HTML]{EFEFEF}
            \textbf{Menace}                         & \textbf{Solution} \\
            \midrule
            Multiples demandes d'introduction       & Les points de rendez-vous peuvent bloquer les requêtes qui ne contiennent pas de jetons d'autorisation, de restreindre le nombre de requêtes recevables ou d'exiger une certaine quantité de calcul pour chaque requête reçue. \\
            \midrule
            Attaquer un point d'introduction        & Désactiver les points de rendez-vous mais ils sont liés à des clés publiques.  \\
            \midrule
            Compromission d'un point d'introduction & Un point de rendez-vous compromis peut inonder le trafic ou empêcher de nouvelles demandes.  \\
            \midrule
            Compromission d'un point de rendez-vous & Trafic chiffré par une clé de session. \\
            \bottomrule
    \end{tabularx}
    \caption{Menaces et défenses dans \acrshort{tor} : points de rendez-vous}
    \label{tab:ad-rdvp}
\end{table}

