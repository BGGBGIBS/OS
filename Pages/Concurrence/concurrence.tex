\section{Concurrence}\label{sec:concurrence}

\subsection{Problématique}
La concurrence en informatique permet d'exécuter plusieurs processus ou threads simultanément, augmentant ainsi la réactivité et l'efficacité des systèmes. 
Toutefois, cette parallélisation introduit de nouveaux défis, notamment les conditions de course, l'exclusion mutuelle et les deadlocks.

\subsection{Mécanismes de communication}
Les processus et threads peuvent communiquer via plusieurs mécanismes :
\begin{itemize}
    \item \textbf{Mémoire partagée} : Utilisation d'une mémoire commune pour échanger des données.
    \item \textbf{Passage de messages} : Envoi de messages explicites entre processus ou threads.
    \item \textbf{Fichiers} : Utilisation de fichiers communs pour la lecture et l'écriture de données.
\end{itemize}

\subsection{Conditions de course}
Les conditions de course surviennent lorsque plusieurs threads accèdent simultanément à des données partagées sans synchronisation adéquate, entraînant des résultats imprévisibles. 
Pour les éviter, des mécanismes de synchronisation tels que les verrous (mutex) et les sémaphores sont utilisés.

\subsection{Sections critiques}
Une section critique est une portion de code qui doit être exécutée par un seul thread à la fois pour éviter les conditions de course. Les solutions incluent :
\begin{itemize}
    \item \textbf{Attente active} : Les threads vérifient continuellement une condition jusqu'à ce qu'elle soit vraie, ce qui peut être inefficace.
    \item \textbf{Blocage} : Les threads se mettent en attente jusqu'à ce qu'ils puissent entrer dans la section critique, ce qui est plus efficace que l'attente active.
\end{itemize}

\subsection{Problèmes classiques}
Les problèmes classiques de la concurrence comprennent :
\begin{itemize}
    \item \textbf{Producteur-consommateur} : Synchronisation entre un producteur qui crée des données et un consommateur qui les utilise.
    \item \textbf{Dîner des philosophes} : Un problème de synchronisation qui illustre les deadlocks et les solutions potentielles, comme l'utilisation de mutex.
\end{itemize}

\subsection{Deadlocks}
Un deadlock survient lorsque plusieurs threads se bloquent mutuellement en attente de ressources. 
Les conditions nécessaires pour un deadlock sont l'exclusion mutuelle, la possession et l'attente, l'absence de préemption, et la formation d'un cycle d'attente.

\subsection{Stratégies pour éviter les deadlocks}
Pour éviter les deadlocks, plusieurs stratégies peuvent être utilisées :
\begin{itemize}
    \item \textbf{Prévention} : Empêcher les conditions nécessaires au deadlock.
    \item \textbf{Évitement} : Utiliser des algorithmes pour éviter d'entrer dans des états de deadlock, comme l'algorithme du banquier.
    \item \textbf{Détection et récupération} : Détecter les deadlocks et prendre des mesures pour les résoudre.
\end{itemize}

\subsection{Conclusion}
La gestion de la concurrence est essentielle pour tirer parti du parallélisme et améliorer les performances des systèmes informatiques. 
Une compréhension approfondie des mécanismes de communication, des conditions de course, des sections critiques et des stratégies de prévention des deadlocks est cruciale pour développer des systèmes robustes
