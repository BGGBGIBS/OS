\section{Mémoire}\label{sec:memoire}
Un programme c'est donc 3 composantes essentielles:
\begin{enumerate}
    \item Code machine: instructions représentant la logique de traitement
    \item Espace de données (RAM): contient les données manipulées par le programme , incluant variables statiques (globales, tableaux) et dynamiques (via malloc)
    \item Stack (espace de travail dédié ecrire et effacer): héberge les données de travail courantes telles que les variables locales et les inforamtions d'appel de fonctions imbriquées pour gérer les retours
\end{enumerate}
Ces éléments sont liés par des adresses représentant la logique du programme (branchements, appels et retours de fonctions, accès aux tableaux et manipulation de stack).

Cablage selon une logique interne
\begin{enumerate}
    \item branchement
    \item Appel retour de fonction
    \item Référence à des tableaux
    \item Manipulation de la stack
\end{enumerate}

Le programme est transformé en mots binaires représentant les instructions et les données via adresses absolues et relatives.

Lors de son exécution, le programme est en mémoire principale et un système de projection des adresses permet de maintenir son éxécution correcte.

Gestion de la mémoire :
\begin{itemize}
    \item Projection : transformer le adresses du programme en adresses physiques disponibles
    \item Protection : restreindre l'accès mémoire pour protéger les données.
    \item Gestion de la mémoire physique : représenter et suivre l'état de la mémoire en fonction des besoins des programmes
\end{itemize}

\subsection{Projection et protection}
\subsubsection{Adressage absolu}
Une adresse absolue désigne un emplacement spécifique dans la mémoire principale, identifiant une instruction ou une donnée précise.

\subsubsection{Adressage relatif}
Une adresse relative représente un décalage par rapport à un point de référence dans la mémoire principale.
\subsubsection{Segmentation}
\subsubsection{Pagination}

\subsection{Gestion de la mémoire physique}
\subsubsection{Bitmpas}
\subsubsection{Listes chaînées}



\section{Mémoire2}\label{sec:memoire2}

\subsection{Problématique}
La gestion de la mémoire est essentielle pour assurer l'efficacité et la stabilité des systèmes informatiques. 
Elle implique la répartition de la mémoire entre les différents processus et la protection de l'espace mémoire alloué. 
Les défis incluent la gestion de la fragmentation, la protection de la mémoire et l'allocation dynamique.

\subsection{Hiérarchie de la mémoire}
La mémoire d'un système informatique est organisée en une hiérarchie :
\begin{itemize}
    \item \textbf{Registres} : Mémoire la plus rapide, utilisée pour stocker les données temporaires.
    \item \textbf{Caches (L1, L2)} : Stockent les données fréquemment utilisées pour réduire les temps d'accès.
    \item \textbf{Mémoire principale (RAM)} : Stocke les données et les programmes en cours d'exécution.
    \item \textbf{Mémoire secondaire (HDD, SSD)} : Utilisée pour le stockage persistant des données.
\end{itemize}

\subsection{Allocation de la mémoire}
L'allocation de la mémoire peut se faire de différentes manières :
\begin{itemize}
    \item \textbf{Allocation fixe} : Les blocs de mémoire ont une taille fixe, ce qui peut conduire à une fragmentation interne.
    \item \textbf{Allocation dynamique} : La mémoire est allouée et libérée au besoin, ce qui nécessite une gestion plus complexe mais réduit la fragmentation.
\end{itemize}

\subsection{Mémoire virtuelle}
La mémoire virtuelle permet à un système d'exécuter des programmes qui nécessitent plus de mémoire que celle disponible physiquement. 
Elle utilise des mécanismes de pagination et de segmentation pour gérer l'allocation et la protection de la mémoire. 
La mémoire virtuelle permet de charger et décharger les portions de mémoire selon les besoins du programme.

\subsection{Protection de la mémoire}
La protection de la mémoire est cruciale pour empêcher un programme d'accéder à la mémoire allouée à un autre programme ou au système d'exploitation lui-même. Les techniques incluent :
\begin{itemize}
    \item \textbf{Protection par segmentation} : Divise la mémoire en segments distincts avec des permissions d'accès spécifiques.
    \item \textbf{Protection par pagination} : Utilise des pages de mémoire avec des permissions d'accès contrôlées par des tables de pages.
\end{itemize}

\subsection{Gestion de la fragmentation}
La fragmentation de la mémoire peut être interne ou externe :
\begin{itemize}
    \item \textbf{Fragmentation interne} : Se produit lorsque des blocs de mémoire ont une taille fixe et que les allocations ne remplissent pas complètement ces blocs.
    \item \textbf{Fragmentation externe} : Se produit lorsque des blocs de mémoire de tailles variables sont alloués et libérés, créant des espaces de mémoire inutilisables.
\end{itemize}

\subsection{Stratégies de gestion de la mémoire}
Plusieurs stratégies peuvent être utilisées pour gérer efficacement la mémoire :
\begin{itemize}
    \item \textbf{Bitmaps} : Utilisent une série de bits pour suivre l'état de chaque unité d'allocation (libre ou allouée).
    \item \textbf{Listes chaînées} : Maintiennent des listes chaînées des blocs de mémoire libres et alloués pour faciliter la gestion dynamique.
\end{itemize}

\subsection{Conclusion}
La gestion de la mémoire est un aspect fondamental des systèmes d'exploitation, impactant directement la performance et la stabilité du système. 
Une compréhension approfondie des techniques d'allocation, de la hiérarchie de la mémoire et des mécanismes de protection est essentielle pour optimiser l'utilisation de la mémoire dans un environnement informatique.
